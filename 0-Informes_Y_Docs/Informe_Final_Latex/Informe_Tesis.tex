% ----------------------------------
% 1-Preambulo.
% ----------------------------------
\documentclass[12pt,a4paper]{article}
\usepackage[spanish]{babel}
\usepackage[T1]{fontenc}
\usepackage{textcomp}
\usepackage{lmodern}
\usepackage[utf8]{inputenc}
\usepackage{graphicx}
\usepackage[procnames]{listings} 	%Para escribir códigos.
% OJO: se agregaron procnames para usarlos en Python (VER).

\usepackage[bottom]{footmisc} 	 	%Para poner las footnote al final de cada página.
\usepackage[hidelinks]{hyperref} 	%Para que el indice pueda ser linkeado.
\usepackage{amssymb}			 	%Para ecuaciones matemáticas.
\usepackage{amsmath}				%Para matrices.
\usepackage{mathtools}
\usepackage{amsfonts} 
\usepackage{verbatim}				%Para usar comentarios.
\parskip 0.1in 						%Distancia parrafos.

%Biliografías:
%\usepackage[style=authoryear]{biblatex}
%\addbibresource{bibliografias.bib}

\usepackage{float} 							%Para que no se muevan las imágenes de lugar.

\usepackage[
  separate-uncertainty = true,
  multi-part-units = repeat
]{siunitx} 									%Para el \SI del +- .

\usepackage[margin=0.984252in]{geometry} 	%Para los márgenes.
\usepackage{subcaption}
\usepackage{appendix} 						%Para los anexos.

% ----------------------------------
% 1.1-Anexos.
% ----------------------------------

%begin anexos
\makeatletter
\def\@seccntformat#1{\@ifundefined{#1@cntformat}  	%"\@seccntformat" es un comando auxiliar.
   {\csname the#1\endcsname\quad}  					%Default.
   {\csname #1@cntformat\endcsname}					%Enable individual control.
}

\let\oldappendix\appendix 							%Guarda la definicion vigente de \appendix
\renewcommand\appendix{%
    \oldappendix
    \newcommand{\section@cntformat}{\appendixname~\thesection\quad}
}
\makeatother
%\renewcommand{\appendixname}{Anexos}
%\renewcommand{\appendixtocname}{Anexos}
%\renewcommand{\appendixpagename}{Anexos}
%end anexos

% ----------------------------------
% 1.2-Para código Python. 
% ----------------------------------
\usepackage{color}
\definecolor{keywords}{RGB}{255,0,90}
\definecolor{comments}{RGB}{0,0,113}
\definecolor{red}{RGB}{160,0,0}
\definecolor{green}{RGB}{0,150,0}
 
\lstset{language=Python, 
        basicstyle=\ttfamily\small, 
        keywordstyle=\color{keywords},
        commentstyle=\color{comments},
        stringstyle=\color{red},
        showstringspaces=false,
        identifierstyle=\color{green},
        procnamekeys={def,class}}

% ----------------------------------
% 1.3-Índice. 
% ----------------------------------

\setcounter{secnumdepth}{3} 		%Para que ponga 1.1.1.1.
\setcounter{tocdepth}{4} 			%Para que añadir las secciones en el Índice.
\usepackage{chngcntr}				%Para que el número de las figuras esten acordes a la sección.
\counterwithin{figure}{section}

\author{
  Calonge, Federico Matias\\
  \text{calongefederico@gmail.com}
}

\title{
  Tesis \\
  \large Automatización de lectura de Currículum Vitae  \\
    para selección de personal}
    
%Para modificar los parrafos y para que se pueda poner subsections:
\makeatletter
\renewcommand\paragraph{\@startsection{paragraph}{4}{\z@}
            {-2.5ex\@plus -1ex \@minus -.25ex}
            {1.25ex \@plus .25ex}
            {\normalfont\normalsize\bfseries}}
\makeatother
\setcounter{secnumdepth}{4} 	%How many sectioning levels to assign numbers to.
\setcounter{tocdepth}{4}    	%How many sectioning levels to show in ToC.
% ----------------------------------
% 2-Documento
% ----------------------------------

\begin{document}

\begin{figure}
  \centering
  \includegraphics[width=0.2\textwidth]{images/undav-logo.png} 	%Incluyendo logo de la Undav.
  \label{fig:undav-logo}
\end{figure}
\maketitle       		%Para generar el título definido arriba.

\cleardoublepage    %Nueva página

\begin{center}
    \Large
    \vspace{0.9cm}
    \textbf{Resumen}
    
\end{center}

En la Tesis de Ingeniería que se presenta, se diseña un \textit{sistema de lectura automática de Curriculum Vitae}. La finalidad del mismo es ayudar al reclutador laboral a elegir a los mejores candidatos para los puestos laborales que tenga disponible mediante una medición de similitud entre textos: Curriculum Vitae de los candidatos por un lado, y keywords que referencian a los puestos laborales por el otro.
El sistema estará desarrollado utilizando el lenguaje de programación Python, y permitirá verificar la teoría desarrollada.

\begin{center}
    \Large
    \vspace{0.9cm}
    \textbf{Abstract}
\end{center}

This Engineering Thesis introduces an \textit{automatic Curriculum Vitae reading system}. The purpose of it is to help the job recruiter to choose the best candidates for the available job positions by means of a measurement of similarity between texts: Curriculum Vitae of the candidates on the one hand, and keywords that refer to job positions on the other. The system will be developed using Python programming language and will allow to verify the developed theory. 

\cleardoublepage    %Nueva página

\tableofcontents 	%Para insertar el índice general.

\cleardoublepage    %Nueva página

\section{Introducción.}
Texto ejemplo. Texto ejemplo. Texto ejemplo. Texto ejemplo. Texto ejemplo. Texto ejemplo.

\subsection{Objetivos del Proyecto.}
Texto ejemplo. Texto ejemplo. Texto ejemplo. Texto ejemplo. Texto ejemplo. Texto ejemplo.

\subsubsection{Objetivo general.}
Texto ejemplo. Texto ejemplo. Texto ejemplo. Texto ejemplo. Texto ejemplo. Texto ejemplo.

\subsubsection{Objetivos específicos.}
Texto ejemplo. Texto ejemplo. Texto ejemplo. Texto ejemplo. Texto ejemplo. Texto ejemplo.

\subsection{Alcance del Proyecto.}
Texto ejemplo. Texto ejemplo. Texto ejemplo. Texto ejemplo. Texto ejemplo. Texto ejemplo.

\subsection{Organización.}
Texto ejemplo. Texto ejemplo. Texto ejemplo. Texto ejemplo. Texto ejemplo. Texto ejemplo.

\section{Reclutamiento laboral en IT.}
Texto ejemplo. Texto ejemplo. Texto ejemplo. Texto ejemplo. Texto ejemplo. Texto ejemplo.

\subsection{Introducción.}
Texto ejemplo. Texto ejemplo. Texto ejemplo. Texto ejemplo. Texto ejemplo. Texto ejemplo.

\subsection{Problemáticas.}
Texto ejemplo. Texto ejemplo. Texto ejemplo. Texto ejemplo. Texto ejemplo. Texto ejemplo.

\subsection{Sistemas de lectura y análisis de CV.}
Texto ejemplo. Texto ejemplo. Texto ejemplo. Texto ejemplo. Texto ejemplo. Texto ejemplo.

\section{Natural Language Processing.}
Texto ejemplo. Texto ejemplo. Texto ejemplo. Texto ejemplo. Texto ejemplo. Texto ejemplo.

\subsection{Introducción.}
Texto ejemplo. Texto ejemplo. Texto ejemplo. Texto ejemplo. Texto ejemplo. Texto ejemplo.

\subsection{Preprocesamiento de textos.}
Texto ejemplo. Texto ejemplo. Texto ejemplo. Texto ejemplo. Texto ejemplo. Texto ejemplo.

\subsection{Similitud entre textos.}
Texto ejemplo. Texto ejemplo. Texto ejemplo. Texto ejemplo. Texto ejemplo. Texto ejemplo.

\subsection{Técnicas para medir Similitud entre textos.}
Texto ejemplo. Texto ejemplo. Texto ejemplo. Texto ejemplo. Texto ejemplo. Texto ejemplo.

\subsubsection{Cosine Similarity.}
Texto ejemplo. Texto ejemplo. Texto ejemplo. Texto ejemplo. Texto ejemplo. Texto ejemplo.

\subsubsection{Word Mover's Distance (WMD).}
Texto ejemplo. Texto ejemplo. Texto ejemplo. Texto ejemplo. Texto ejemplo. Texto ejemplo.

\subsection{Algoritmos de vectorización.}
Texto ejemplo. Texto ejemplo. Texto ejemplo. Texto ejemplo. Texto ejemplo. Texto ejemplo.

\subsubsection{TF-IDF.}
Texto ejemplo. Texto ejemplo. Texto ejemplo. Texto ejemplo. Texto ejemplo. Texto ejemplo.

\subsubsection{Word Embeddings.}
Texto ejemplo. Texto ejemplo. Texto ejemplo. Texto ejemplo. Texto ejemplo. Texto ejemplo.

\paragraph{¿Cómo entrenar Word Embeddings?}
Texto ejemplo. Texto ejemplo. Texto ejemplo. Texto ejemplo. Texto ejemplo. Texto ejemplo.

\section{Desarrollo.}
Texto ejemplo. Texto ejemplo. Texto ejemplo. Texto ejemplo. Texto ejemplo. Texto ejemplo.

\subsection{Esquema del sistema.}
Texto ejemplo. Texto ejemplo. Texto ejemplo. Texto ejemplo. Texto ejemplo. Texto ejemplo.

\subsection{Set de Datos.}
Texto ejemplo. Texto ejemplo. Texto ejemplo. Texto ejemplo. Texto ejemplo. Texto ejemplo.

\subsubsection{Curriculum Vitae.}
Texto ejemplo. Texto ejemplo. Texto ejemplo. Texto ejemplo. Texto ejemplo. Texto ejemplo.

\subsubsection{Descripciones Puestos Laborales.}
Texto ejemplo. Texto ejemplo. Texto ejemplo. Texto ejemplo. Texto ejemplo. Texto ejemplo.

\subsection{Medición de similitud entre textos.}
Texto ejemplo. Texto ejemplo. Texto ejemplo. Texto ejemplo. Texto ejemplo. Texto ejemplo.

\subsubsection{FALTA: Pasos de las pruebas que hice: "Mejorando…" / "Implementación de…" / "Problemas al…".}
Texto ejemplo. Texto ejemplo. Texto ejemplo. Texto ejemplo. Texto ejemplo. Texto ejemplo.

\subsection{Resultados Obtenidos.}
Texto ejemplo. Texto ejemplo. Texto ejemplo. Texto ejemplo. Texto ejemplo. Texto ejemplo.

\subsection{Conclusiones.}
Texto ejemplo. Texto ejemplo. Texto ejemplo. Texto ejemplo. Texto ejemplo. Texto ejemplo.

\subsection{Agregando funcionalidades.}
Texto ejemplo. Texto ejemplo. Texto ejemplo. Texto ejemplo. Texto ejemplo. Texto ejemplo.

\subsubsection{Base de datos.}
Texto ejemplo. Texto ejemplo. Texto ejemplo. Texto ejemplo. Texto ejemplo. Texto ejemplo.

\subsubsection{Framework Web.}
Texto ejemplo. Texto ejemplo. Texto ejemplo. Texto ejemplo. Texto ejemplo. Texto ejemplo.

\subsubsection{Roles y Usuarios.}
Texto ejemplo. Texto ejemplo. Texto ejemplo. Texto ejemplo. Texto ejemplo. Texto ejemplo.

\subsubsection{Manejo de los datos.}
Texto ejemplo. Texto ejemplo. Texto ejemplo. Texto ejemplo. Texto ejemplo. Texto ejemplo.

\paragraph{Modelado.}
Texto ejemplo. Texto ejemplo. Texto ejemplo. Texto ejemplo. Texto ejemplo. Texto ejemplo.

\paragraph{Filtrado.}
Texto ejemplo. Texto ejemplo. Texto ejemplo. Texto ejemplo. Texto ejemplo. Texto ejemplo.

\paragraph{Visualización.}
Texto ejemplo. Texto ejemplo. Texto ejemplo. Texto ejemplo. Texto ejemplo. Texto ejemplo.

\subsection{Caso de Uso.}
Texto ejemplo. Texto ejemplo. Texto ejemplo. Texto ejemplo. Texto ejemplo. Texto ejemplo.

\subsection{Limitaciones del sistema.}
Texto ejemplo. Texto ejemplo. Texto ejemplo. Texto ejemplo. Texto ejemplo. Texto ejemplo.
 
\section{Próximos pasos.}  
Texto ejemplo. Texto ejemplo. Texto ejemplo. Texto ejemplo. Texto ejemplo. Texto ejemplo.

\section{Glosario.}
Texto ejemplo. Texto ejemplo. Texto ejemplo. Texto ejemplo. Texto ejemplo. Texto ejemplo.

\section{Anexo.}
Texto ejemplo. Texto ejemplo. Texto ejemplo. Texto ejemplo. Texto ejemplo. Texto ejemplo.

\section{Bibliografía.}
Texto ejemplo. Texto ejemplo. Texto ejemplo. Texto ejemplo. Texto ejemplo. Texto ejemplo.

\section{Agradecimientos.}
Texto ejemplo. Texto ejemplo. Texto ejemplo. Texto ejemplo. Texto ejemplo. Texto ejemplo.

\end{document}


%Para poner pie de paginas --> \footnote{Un conjunto de datos, conocido también como dataset, es una colección de datos habitualmente tabulada.})

%Para poner en Las citas del anexo --> información\cite{iot}. 

